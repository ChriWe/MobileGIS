\chapter{Einleitung}
\label{chp:einleitung}

Im Rahmen der LVA Mobile GIS Anwendung wird eine Applikation entwickelt. Dabei sollen alle während des Semesters gelernten elementaren Bestandteile einer mobilen Anwendung in Betracht gezogen werden. Es soll, unter Verwendung von gängigen Web-Technologien, speziell auf die Bedürfnisse von mobilen Anwendungen eingegangen werden.

\section{Entwicklungsziele}
\label{sec:ziele}

Auf Basis der im Semesterstoff erarbeiteten Lehrziele, soll die fertige Lösung folgende elementare Komponenten implementieren.

\begin{itemize}
  \setlength{\itemsep}{1pt}
  \setlength{\parskip}{0pt}
  \setlength{\parsep}{0pt}
  
  \item Verwendung einer gängigen Javascript-API für responsive mobile Design
  \item Verwendung der Overpass-API für die Abfrage von OSM-Daten
  \item Feature-Implementierung unter der Verwendung der Smartphone-Sensorik
  \item Datenbankanbindung an CouchDB
  \item Offline-Synchronisation der gespeicherten Daten
  
\end{itemize}

Des weiteren wurde bei der Implementierung stets auf folgende Unterpunkte geachtet.

\begin{itemize}
  \setlength{\itemsep}{1pt}
  \setlength{\parskip}{0pt}
  \setlength{\parsep}{0pt}
   
  \item Implementierung des MVC-Pattern
  \item Modularisierte Architektur
  \item Skallier- und Erweiterbarkeit
  \item Konfigurierbarkeit
  \item Automatisiertes Assembling der Android-APK Datei
  
\end{itemize}

\section{Methodik}
\label{sec:methodik}



Die Entwicklungsziele bilden die Grundlage und Rahmenbedingungen zur Implementierung für den Entwurf einer Android-Applikation namens 'Sighted!'. Die Methodik zur Entwicklung der vorgestellten Problemstellung umfasst folgende Punkte:

\begin{itemize}
  \setlength{\itemsep}{1pt}
  \setlength{\parskip}{0pt}
  \setlength{\parsep}{0pt}
  
  \item Aufsetzen einer Hybriden-Android Applikation, die den Zugriff auf hardwarenahe Sensorikfunktionen mittels Javascript ermöglicht.
  \item Konfiguration des Assembling-Prozesses der Applikation mittels Gradle
  \item Aufsetzen der grundlegenden Projektstruktur der Web-Applikation
  \item Konfiguration der Web-Applikation mittels Grunt
  \item Einbindung von gängigen Web-Frameworks: jquery (mobile), bootstrap, requireJS, ol3, pouchdb
  \item Implementierung des Grundgerüsts der Web-Applikation
  \item Implementierung eines Wrappers für die Verwendung der Overpass-API
  \item Implementierung eines Wrappers für die Verwendung der pouchDB-API
  \item Implementierung der Anwendungslogik und Design der Anwendung 

\end{itemize}